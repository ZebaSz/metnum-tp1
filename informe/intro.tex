\section{Introducción}

En este informe se detalla el trabajo realizado sobre el análisis
de los métodos numéricos para la técnica computacional de fotometría estéreo,
que permite generar digitalizaciones de objetos 3D basados en imágenes.

Para este trabajo, contamos con conjuntos de imágenes 2D de varios objetos
tridimensionales. Estas imágenes muestran un objeto desde el mismo ángulo
pero con iluminación de diversas fuentes. A su vez, contamos con una figura
esférica que comparte las fuentes de iluminación, para utilizar para obtener
información sobre estas luces. Por último, cada conjunto de imágenes de una
misma figura cuenta con una \textit{máscara}, que marca las secciones de la
imagen que corresponden al objeto en cuestión, a modo de reducir el ruido
o el peso del cómputo.

En base a estas imágenes y a la técnica mencionada, se intenta reconstruir
y medir la posición y distancia de los objetos con respecto a la cámara.
