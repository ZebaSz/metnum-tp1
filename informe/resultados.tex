\section{Futura experimentación}

A lo largo del trabajo encontramos algunos detalles que consideramos
fuera del marco de este trabajo, ya sea por complejidad o falta de tiempo.
Sin embargo, nos gustaría mencionar algunas de las cosas que destacamos como
posible experimentación futura:

\begin{itemize}
	\item La calibración utilizada para este trabajo es muy sencilla, pero como
	se demostró es bastante sensible y falible.

	Entre otras cosas, consideramos agregar una lógica de area a la calibración,
	en lugar de hacerla puntual: la misma determinaría el area más iluminada,
	reduciendo la probabilidad de ser afectado por un punto de intensidad
	inusual. Luego, se buscaría el máximo de este area, o se tomaría el
	punto central, y se tomaría la normal de este para determinar la
	dirección de la luz.

	\item Si bien calculamos todas las direcciones de luz, en ningún momento
	utilizamos más de 3, y las mismas están actualmente fijas en el código.

	Nos gustaría poder determinar dinámicamente el conjunto de luces con mayor
	probabilidad de producir un resultado confiable. Esto requiere determinar
	alguna heurística, ya sea:

	\begin{itemize}
		\item A nivel general, utilizando la distancia de los puntos más iluminados/el
		ángulo entre las direcciones de la luz: esto requiere un análisis sobre
		qué distancias/ángulos son los más eficientes, que nosotros suponemos no
		deben estar demasiado cerca para que cada luz provea información nueva,
		pero tampoco tan alejados porque se corre el riesgo de introducir ruido;

		\item A nivel individual, dependiendo de la figura: esto es mucho más complejo
		para el cual no formulamos una teoría, porque puede variar mucho dependiendo
		de los detalles de la figura, y podríamos terminar resaltando datos poco
		relevantes y perdiendo la forma general de la figura.
	\end{itemize}

	\item En nuestro trabajo utilizamos 3 luces para generar el campo normal.
	Si bien, asumiendo que las luces fueron elegidas correctamente, esto provee
	información suficiente para reconstruir el objeto, nos gustaría saber
	cuánto ruido o cuánta precisión adquiere el sistema al agregar más fuentes
	de iluminación.

	Sin embargo, nuestra intuición nos dice que esto no aportaría nada al sistema,
	y estamos convencidos que cualquier mejora posible (si alguna)
	no justificaría el costo de cómputo de aumentar la dimensión del sistema.
\end{itemize}